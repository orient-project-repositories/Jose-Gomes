\documentclass[a4paper]{paper}

% better use of an 4a page, lesser margins
\usepackage[a4paper,left=2cm,right=2cm,top=2cm,bottom=2cm]{geometry}

\usepackage[utf8]{inputenc} % utf8 allows portuguese characters
\usepackage{amsmath,amssymb,amsfonts,amsthm,bm} % help maths
\usepackage{xcolor} % allow writing in color e.g. \textcolor{red}{see here!}

\title{Example of writing one very simple document using LaTeX}
\author{author1 author2 author3}


\begin{document}
\maketitle

\begin{abstract}
One can optionally include an abstract section in a paper. There is no limit on the size of the abstract, but usually it comprises just some few paragraphs and in many cases is one single paragraph with half a dozen lines.
\end{abstract}

After starting a document, one can build various sections and sub-sections.

\section{Introduction}

\subsection{Related work}

\subsection{Problem description}

\section{Our proposal}

\section{Results}

\section{Conclusions and future work}

\end{document}
