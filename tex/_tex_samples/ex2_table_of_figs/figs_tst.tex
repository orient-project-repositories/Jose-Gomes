\documentclass{article}
\usepackage{graphicx} % to includegraphics
\usepackage{float} % to position figure H

\title{Title of the Paper}
\author{author}

\begin{document}
\maketitle
\begin{abstract}
This abstract is very abstract
\end{abstract}

This is a fast demonstration of making composite figures. See Fig.\ref{fig:figs_tst}.
Notice that in order to place a number of EPS files side by side, a tabular environment has been used.
One figure can hold any number of tabular environments inside, and in this way make a composite figure of a variable number of figures in each line.

\begin{figure}[H]
	\centering
		\begin{tabular}{ccc}
		   \includegraphics[width=3cm]{sph_proj.eps} &
		   \includegraphics[width=3cm]{sph_proj.eps} &
		   \hspace{1cm} \includegraphics[width=3cm]{sph_proj.eps} \\
		   (a) & (b) & (c) \\
		\end{tabular}
	\caption{text of the caption (a) is ... (b) illustrates ... (c) shows ...}
	\label{fig:figs_tst}
\end{figure}

\end{document}
