%
%   Disserta��o: (colocar aqui nome pr�prio + apelido)
%		July 2018
%

% Sites / docs to check:
%
% MEEC:
% https://fenix.tecnico.ulisboa.pt/cursos/meec/documentos-e-regulamentos
% https://fenix.tecnico.ulisboa.pt/downloadFile/845043405460433/DMEEC201803.pdf
%
% IST:
% https://graduacao.tecnico.ulisboa.pt/alunos/dissertacao-de-mestrado/
% https://graduacao.tecnico.ulisboa.pt/files/sites/54/guia-de-preparacao-da-dissertacao-1516.pdf
% https://tecnico.ulisboa.pt/pt/recursos/documentos-importantes/


% --------------------------------------------------------- CONFIG:

\documentclass[a4paper,12pt]{book}

%\usepackage[latin1]{inputenc}   % incluir uma imagem postscript
%\usepackage[ansinew]{inputenc} 	% acentua��o autom�tica
\usepackage[portuguese,english]{babel}	% titles in last chosen language
\usepackage[ansinew]{inputenc}
\usepackage[T1]{fontenc}
\usepackage{times}    					% poe letra bonita em PDF
%\usepackage{indentfirst}		    % indenta o primeiro paragrafo de cada sec��o

%\usepackage[dvips]{graphicx}
\usepackage{graphicx}
\graphicspath{{./}{./figs/}{../figs/}} % share figs using ../figs/
\usepackage{subfigure}					% figuras paralelas ou seguidas com mesma legenda
\usepackage{subcaption} 

%\usepackage{epstopdf}
\usepackage{color}

\usepackage{amssymb}
\usepackage{amsmath}
\usepackage{amsthm}          
%\usepackage[colorlinks=false, pdfstartview=FitV, linkcolor=blue, citecolor=blue, urlcolor=blue,breaklinks=true]{hyperref}
%\usepackage{subeqn}							% subnumerar equa��es

\usepackage{url}
\usepackage{glossaries}

%\usepackage{hyperref}   				% poe bookmarks no ficheiro pdf
%\PassOptionsToPackage{linktocpage}{hyperref}
\usepackage[linktocpage=true]{hyperref}

\usepackage{setspace}						% pacote de espa�amento
\usepackage{datetime}
\onehalfspacing									% espa�amento de 1,5 entre linhas
%\doublespacing			 

%\vspace{.5cm} 									% como por um espa�amento vertical

\textwidth 			=   16cm 				% largura do texto na folha
\textheight 		=   22cm 				%
\oddsidemargin 	= 	0cm 				% margens na folha
\evensidemargin =   0cm 				%
\marginparwidth =   0pt 				% margem de paragrafos
%\usepackage[top=25mm, bottom=25mm, left=25mm, right=25mm]{geometry}

\usepackage{fancyhdr} 					% cabe�alhos

\pagestyle{fancy}								% cabe�alhos do cap�tulo e sec��o em min�sculas
\renewcommand{\chaptermark}[1]{\markboth{#1}{}}
\renewcommand{\sectionmark}[1]{\markright{\thesection\ #1}}
\fancyhf{} % apagar as configura��es actuais
\fancyhead[LE,RO]{\bfseries\thepage}
\fancyhead[LO]{\bfseries\rightmark}
\fancyhead[RE]{\bfseries\leftmark}
\renewcommand{\headrulewidth}{0.5pt}
\renewcommand{\footrulewidth}{0pt}
\addtolength{\headheight}{4.5pt} % fazer espa�o para o risco
\fancypagestyle{plain}{%
	\fancyhead{} % Tirar cabe�alhos de p�gina vazias
	\renewcommand{\headrulewidth}{0pt} % e o risco
}


%\includeonly{sec2_meth1}
%\includeonly{sec3_meth2}
%\includeonly{sec4_experiments}


% --------------------------------------------------------- MAIN:

\begin{document}
%-------------------------------------------
% cover page
%-------------------------------------------

\thispagestyle{empty}

% -- Option 1/2 large letters:
\newcommand{\myfontA}{\LARGE} \newcommand{\myfontB}{\Large} \newcommand{\myfontC}{\large} 
% -- Option 2/2 large letters but not so large:
%\newcommand{\myfontA}{\Large} \newcommand{\myfontB}{\large} \newcommand{\myfontC}{\normalsize} 

\singlespace

% Logo -------------------------------------------

$ $ \vspace{-3cm}

\begin{minipage}[t]{5cm}
 %\includegraphics[width=\linewidth]{istlogo2.eps} % high res
 \includegraphics[width=5cm]{istlogo3.eps} % IST logo 2013
\end{minipage}

%\begin{minipage}[t]{12.5cm}
%\centering
% \vspace*{-1.5cm}
% {{\myfontC \bf UNIVERSIDADE T�CNICA DE LISBOA}\\
%  {\myfontC \bf \vspace{0.2cm} INSTITUTO SUPERIOR T�CNICO}}
%\end{minipage}

\vspace*{-0.5cm} % \hspace*{-1.3cm}
\begin{center}
$ $ \\
\end{center}


%% Boneques ------------------------------------------------

%\begin{figure}[hbtp]
% \centerline{\includegraphics[height=4cm]{cover.eps}}
% \vspace{4mm}
%\end{figure}

% alternative to the Boneques:
\vfill

%% Title --------------------------------------------------
\onehalfspace
\begin{center}
  \myfontA
  {\bf Title of the Thesis}
\end{center}
%\vspace{7mm}
\singlespace


%% Author ---------------------------------------------------
\vfill

\begin{center}
  % place here the full name, i.e first, middle and last names
  {\myfontB \bf Jos\'e Pedro Ribeiro Gomes} \\
  %(Licenciado)
\end{center}


%% Subject -------------------------------------------------
\vfill

%% Use the following if thesis title/contents is in Portuguese
%\onehalfspace
%\begin{center}
% {Disserta��o para a obten��o do Grau de Mestre em}\\
% $ $ \\
% {Engenharia Electrot�cnica e de Computadores}
%\end{center}
%\singlespace

%% Use the following if thesis title/contents is in English
\onehalfspace
\begin{center}
 {\myfontC Thesis to obtain the Master of Science Degree in}\\
 $ $ \\
 {\myfontB \bf Electrical and Computer Engineering}
\end{center}
\singlespace

%% Supervisor & juri ----------------------------------------------
\vfill

\onehalfspace

%% --- T�tulo da tese em Portugu�s => j�ri em Portugu�s
%\begin{center}
%{\myfontB \bf J�ri}
% $ $ \\
%\myfontC
% $ $      & {} \\
% { Presidente: Professor } \\
% { Orientador: Professor Jos� Ant�nio da Cruz Pinto Gaspar} \\
% { Co-orientador: Doutor } \\
% { Vogal: Professor } \\
%% { Vogal: } \\ % in case of more than one
%\end{center}

\begin{center}
{\myfontB \bf Supervisor} \\
%{\myfontB \bf Supervisors} \\
$ $ \\
\myfontC
 { Professor Jos\'e Ant\'onio da Cruz Pinto Gaspar } \\
 { Professor Alexandre Jos\'e Malheiro Bernardino}
 %{ Professor  } \\
\end{center}

\vfill

%% --- English title => committee in English
\begin{center}
{\myfontB \bf Examination Committee} \\
$ $ \\
\myfontC
 { Chairperson: \emph{to be filled later} } \\
 { Supervisor: Professor Jos\'e Ant\'onio da Cruz Pinto Gaspar} \\
 { Member: \emph{to be filled later} } \\
% {Member: } \\ % in case of more than one
\end{center}

\singlespace


%% Date ----------------------------------------------------
\vfill

\onehalfspace \centerline{\myfontB \bf \monthname \, \the\year} \singlespace

\pagebreak \thispagestyle{empty} $ $ \pagebreak
 \onehalfspacing 

\cleardoublepage \pagenumbering{roman}%


% NOTE1: An acknowledgments subsection is read by everyone, including people that knows nothing about the thesis AND no one revises it (even the supervisors).
% NOTE2: In order to include and acknowledgments subsection, please uncomment the next lines.
%
%\section*{Agradecimentos}
%\addcontentsline{toc}{section}{Agradecimentos}
%
%� com bastante satisfa��o que escrevo esta sec��o por finalmente ter vencido uma meta importante...


% -------------------------------------------------------------------------------

%\section*{Resumo}
%\addcontentsline{toc}{section}{Resumo}
%
%(Resumo em Portugu�s)
%
%
%\vspace{1cm}\noindent\textbf{Palavras chave:}
%C�mara pan tilt zoom, ...


\section*{Abstract}
\addcontentsline{toc}{section}{Abstract}

Although it is clear the intrinsic advantages of pan-tilt cameras with respect to fixed cameras, modern video surveillance systems are still based on networks of fixed cameras...


\vspace{1cm}\noindent\textbf{Keywords:}
Pan and Tilt Zoom Camera, ...


\tableofcontents % main index
%\listoffigures
%\listoftables

\cleardoublepage \pagenumbering{arabic}

\section{Introduction}

Many modern video surveillance systems encompass pan-tilt cameras due to the flexibility they provide in selecting the fields-of-view, as compared to using just fixed cameras. Although patent the great potential of using the pan-tilt cameras, one has to design pan and tilt controllers whose effectiveness directly impacts on the surveillance performance.
%
The work described within this dissertation ...


\subsection{Related Work} 

Related work, e.g. a simple to use camera calibration toolbox \cite{Bouguet}.

Related work, on another relevant subject ...


\subsection{Problem Formulation} 

% Identify one or two main difficulties in state of the art knowledge
% and briefly state how to approach those difficulties

Surveillance with pan-tilt cameras involves not only video processing but also controlling the pan and tilt angles. In this work ...


\subsection{Report Structure} 

Section 1 introduces the problem to approach in the thesis, in particular presents a short discussion on the state of the art on ... 
%
Section 2 presents ...
%
Section 3 describes ...
%
Section 4 provides an overview of the different experiments executed as well as the results attained.
%
Section 5 summarizes the work performed and highlights the main achievements in this work. Moreover, this section proposes further work to extend the activities described in this document.

The work described hereafter was partially published in (put ref here).

\chapter{Method 1}
\label{sec:meth1}

There is a large variety of segmentation algorithms...

\chapter{Method 2}
\label{sec:meth2}

One advantage of pan-tilt cameras with respect to fixed cameras is the capability of augmenting the surveyed area without adding extra complexity to the system...

\chapter{ Experiments }

This chapter describes the experiments performed to validate the methodologies presented and introduced in previous chapters. Four experiments were performed addressing the ...


\section{ Conclusion and Future Work}

The work described in this thesis...

Two main technical constraints were originally identified in this work, (i) ... and (ii) ...

In particular we studied the ...

In future work we plan to extend the framework for multiple pan-tilt-camera scenarios, where questions like how to define regions of interest for a multiple pan-tilt-camera scene? or how to share information between the cameras? would arise.


\appendix
\include{secx0_app}
\include{secx1_app}

\bibliographystyle{plain}
\bibliography{secz}

\end{document}
