\chapter{Introduction}

\epigraph{``Sic parvis magna" (Greatness from small beginnigs)}{--- \textup{Sir Francis Drake}}


Many modern video surveillance systems encompass pan-tilt cameras due to the flexibility they provide in selecting the fields-of-view, as compared to using just fixed cameras. Although patent the great potential of using the pan-tilt cameras, one has to design pan and tilt controllers whose effectiveness directly impacts on the surveillance performance.
%
The work described within this dissertation ...


\section{Related Work} 

Related work, e.g. a simple to use camera calibration toolbox.

Related work, on another relevant subject ...


\section{Problem Formulation} 

% Identify one or two main difficulties in state of the art knowledge
% and briefly state how to approach those difficulties

Surveillance with pan-tilt cameras involves not only video processing but also controlling the pan and tilt angles. In this work ...


\section{Thesis Structure} 

Chapter 1 introduces the problem to approach in the thesis, in particular presents a short discussion on the state of the art on ... 
%
Chapter 2 presents ...
%
Chapter 3 describes ...
%
Chapter 4 provides an overview of the different experiments executed as well as the results attained.
%
Chapter 5 summarizes the work performed and highlights the main achievements in this work. Moreover, this chapter proposes further work to extend the activities described in this document.

The work described hereafter was partially published in (put ref here).
